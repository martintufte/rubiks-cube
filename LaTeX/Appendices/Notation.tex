\section{Definitions}

\subsection*{Notation}

\textbf{Def.} The \underline{Rubik's Cube group} $G$ represents the mechanical puzzle invented by Ernő Rubik in 1974. Each element of $G$ can be constructed using a sequence of \underline{face turns} \{\seq{R, L, U, D, F, B}\}, which moves the corresponding faces 90 degrees clockwise. Other moves include \underline{slice turns} \{\seq{M, E, S}\} and \underline{rotations} \{\seq{x, y, z}\}.

\textbf{Def.} The \underline{identity element} of $G$ is denoted as $\seq{I}$ and corresponds to the empty move.

\textbf{Def.} A \underline{sequence} \seq{A} is a concatenation of moves, e.g. \seq{A = R U F} and defines a permutation of $G$.

\textbf{Def.} A permutation \seq{P} has \underline{inverse} \seq{P'} := \seq{P}$^{-1}$ and \underline{double} \seq{P2} := \seq{P}$^{2}$. Note that for the standard moves, including slices and rotations, \seq{P P' = P' P = I}, \seq{P2 P' = P' P2 = P}, \seq{P2 P = P P2 = P'} and \seq{P2 P2 = I}. If the permutation comes from a sequence \seq{A}, then the inverse permutation \seq{A'} is defined such that \seq{A A' = I}. E.g. for \seq{A = F L2 B'}, then \seq{A' = B L2 F'}.

\textbf{Def.} A \underline{commutator} is a sequence of the form \seq{A B A' B'} which is useful for affecting just a few pieces on the cube while leaving everything else untouched. For two arbitrary sequences \seq{A} and \seq{B}, the commutator will affect only the parts of the cube that are in the intersection of \seq{A} and \seq{B}. If \seq{A} and \seq{B} are completely disjoint and affect completely different parts of the cube, then obviously \seq{A B A' B'} will do exactly nothing.

\textbf{Def.} A \underline{conjugate} is a sequence of the form \seq{A B A'}. The sequence \seq{A} is usually called the setup because it is applied at the start, before \seq{B} and is reversed at the end, after \seq{B}.

\subsection*{More sequence manipulations}

\textbf{Def.} During a reconstruction, the part which is solved on the inverse is written using parenthesis. E.g. \seq{(L U2 L')}. Then this sequence is used to solve some part on the inverse of the scramble. This can be added to a normal solve using pre-moves before applying the scramble. A lot of properties are conserved on the inverse scramble, such as the number of oriented edges and corners and the number of pairs.


\subsection*{Comments}

The comments in the reconstructions are useful to say which step is solved during that step. Useful comments include 2x2x2, 2x2x3, F2L, 1. pair, EO, F2L-1, DR, Setup to 4c4e, HTR, OLL, PLL, AUF, CO, Insertions, Skeleton, etc. Also useful to count the number of moves. E.g. (6-1/21), means that the sequence adds 6 moves, cancels 1 move with cumulative count 21.

For skeletons, the remaining pieces are usually labelled according to the number of pieces in the remaining cycles, with naming convention 'e' for edge, 'c' for corners, 'x' for centers and 'p' for edge/corner pairs. The comment '// 3c3e' means that the sequence of moves solves all pieces except for 3 corners and 3 edges. Another example is '2e2e', which solves all expect for two edges which must be swapped and two other edges which must be swapped.  

\subsection*{Metrics for measuring length of sequences}

\textbf{Def.} Using the \underline{Quarter-Turn-Metric} face turns count as 1, with double counting as 2, slices count as 2, with double counting as 4 and rotations count as 0.

\textbf{Def.} Using the \underline{Half-Turn-Metric} face turns, including doubles, count as 1, slices count as 2 and rotations count as 0.

\textbf{Def.} Using the \underline{Slice-Turn-Metric} face turns and slices, including doubles, count as 1 and rotations count as 0.

\textbf{Def.} Using the \underline{Axis-Turn-Metric} face turns and slices, including doubles and moving opposite faces, count as 1 and rotations count as 0.

\section{Color Scheme and orientation}
\SolvedCube

I used the standard color scheme and orientation (green on front and white on top).
\begin{center}
\DrawCubeIcon{3}{1}
\end{center}